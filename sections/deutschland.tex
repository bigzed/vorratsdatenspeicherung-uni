%!TEX root = ../main.tex

\section{Umsetzung in Deutschland}

  \begin{frame}
  \frametitle{Gesetz zur Neuregelung der Telekommunikationsüberwachung}
    \begin{itemize}
      \item 
        regelte vom 1. Januar 2008 bis 2. M"arz 2010 die Vorratsdatenspeicherung
      \item 
        entgegen der EU-Richtlinie waren ab 1. Januar 2009 auch nicht kommerzielle Dienste zur Speicherung verpflichtet
      \item 
        Vorratsdatenspeicherung bei nicht Verpflichtung bestraft mit Geldbu"se bis 10000 Euro (siehe \S 149 Abs. 1 Nr 17 TKG)
    \end{itemize}
  \end{frame}

  \begin{frame}
    \frametitle{Nutzung der Daten}
    \begin{itemize}
      \item Verfolgung von Straftaten
      \item Abwehr von erheblichen Gefahren f"ur die "offentliche Sicherheit
      \item Erf"ullung der Aufgaben von Verfassungsschutzbeh"orden, Bundesnachrichtendiensten und Milit"arischen Abschirmdienstes
      \item Asuk"unfte "uber Identit"at von Telekomunikatons- und Internetnutzern nach \S 113
      \item Urheberrechtsverletzungen im Internet
    \end{itemize}
  \end{frame}

  \begin{frame}
    \frametitle{Verfassungsbeschwerde}
    \begin{itemize}
      \item 31. Dezember 2007 wurde vom Arbeitskreis Vorratsdatenspeicherung initiierte Sammel-Verfassungsbeschwerde eingereicht
      \item insgesamt 34939 Beschwerdef"uhrer
      \item 11. M"arz 2008 einstweilige Verf"ugung, Nutzung der Daten nur noch bei schweren Straftaten
      \item Bundesregierung zu Bericht bis 1. September "uber praktische Ausirkungen verpflichtet
    \end{itemize}
  \end{frame}

  \begin{frame}
    \frametitle{Urteil}
    \begin{itemize}
      \item Vorschriften zur Vorratsdatenspeicherung sind verfassungswidrig
      \item Gesetz in seiner Form verst"o"st gegen Art. 10 Abs. 1 GG
      \item nicht generell Unvereinbar mit Gesetz
      \item Daten sollten dezentral gespeichert und besonders gesichert werden
      \item Beh"orden nur bei genau spezifizierten F"allen zugriff gew"ahren
      \item Ermittlung der IP-Adresse selbst bei Ordungswidrigkeiten zul"assig
    \end{itemize}
  \end{frame}
