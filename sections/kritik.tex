%!TEX root = ../main.tex
\section{Kritik}
  \subsection*{Unverhältnismäßige geringe Nutzung}
    \begin{frame}<beamer>{Unverhältnismäßige geringe Nutzung}
      \begin{itemize}
        \item
          Abschreckungfaktor ist nicht vorhanden.
        \item
         Umgehungsmöglichkeiten sind auch für Laien möglich.
        \begin{itemize}
          \item TOR-Netzwerk
          \item alternative Emaildienste
          \item bei SMS auf Alternativen umsteigen (zb. Whatsapp)
        \end{itemize}
        \item Durch Vorratsdatenspeicherung hätte weder 9/11 als auch die Attentate in Großbritannien 2005 verhindert werden können
      \end{itemize}
    \end{frame}

    \begin{frame}<beamer>{Straftaten in "Osterreich Vergleich}
      \begin{quote}
        Dennoch zeigten die Daten der österreichischen Vorratsdatenspeicherung, daß die angeblich schwersten Straftaten, bei denen die Datensätze abgerufen werden sollten, in Wahrheit in erster Linie Diebstahlsdelikte waren, außerdem Stalking. Bei Organisierter Kriminalität oder Taten, die als Terror definiert sind, wurden die zwangsweise gespeicherten Daten in genau null Fällen verwendet.

        \attrib{\url{http://www.ccc.de/de/vorratsdatenspeicherung }}
      \end{quote}
    \end{frame}

\begin{frame}<beamer>{Schwere Strafdaten in Deutschland Statstik}
\begin{itemize}
        \item Schwere Strafdaten in Deutschland Statstik
        \includegraphics[height=1\textheight]{sections/img/schwere_verbrechen_in_DE.png}
    \end{itemize}
    \end{frame}
    
    \begin{frame}<beamer>{Schwere Verbrechen in Deutschland Aufklärung Statistik}
\begin{itemize}
        \item Schwere Verbrechen in Deutschland Aufklärung Statistik
        \includegraphics[height=1\textheight]{sections/img/aufklaerung_in_DE.png}
    \end{itemize}
    \end{frame}
      \begin{frame}<beamer>{Internetstrafdaten in Deutschland Statistik}
\begin{itemize}
        \item Internetstrafdaten in Deutschland Statistik
        \includegraphics[height=1\textheight]{sections/img/internet_delikte_in_DE.png}
    \end{itemize}
    \end{frame}
          \begin{frame}<beamer>{Internetstrafdaten in Deutschland Aufklärung Statistik}
\begin{itemize}
        \item Internetstrafdaten in Deutschland Aufklärung Statistik
        \includegraphics[height=1\textheight]{sections/img/aufklaerung_internetdelikte_DE.png}
    \end{itemize}
    \end{frame}
              \begin{frame}<beamer>{Aufklärungsquote Allgmein}
\begin{itemize}
        \item Aufklärungsquote Allgmein
        \includegraphics[height=1\textheight]{sections/img/aufklaerung.png}
    \end{itemize}
    \end{frame}
              \begin{frame}<beamer>{Interpretation der Statistik des Bundeskriminalamtes}
\begin{itemize}
        \item Die VDS brachte keine Erhöhte Aufklärungsquote
        \item Es konnte keine Senkung der Kriminalitätsrate festgestellt werden
        \item Die Aufklärungsrate der Internetstraftaten sank im Zeitraum der VDS
    \end{itemize}
    \end{frame}



  \subsection*{Missbrauch und Irrtumsrisiko}
    \begin{frame}<beamer>{Missbrauch und Irrtumsrisiko}
      \begin{itemize}
        \item
          Telekommunikationsdaten haben eine sehr hohe Aussagekraft
      \begin{itemize}
         \item mit Methoden von Data-mining können scheinbar belanglose Daten eine hohe Aussagekraft bekommen
      \end{itemize}
        \item
          Rückschlüsse auf die gesamte Lebensituation möglich
 \item viele Interessensgruppen haben Interesse an den sensiblen Daten
          \begin{itemize}
         \item Behörden/Staat
         \item politische Gruppierungen
         \item Personen aus Privatenumfeld
      \end{itemize}
 
      \end{itemize}
    \end{frame}

  \subsection*{Juristische Argumente}
    \begin{frame}<beamer>{Juristische Argumente}
      \begin{itemize}
        \item Verstoß gegen Europarecht
           \begin{itemize}
         \item Verstoß gegen Gemeinschaftsgrundrechte
      \end{itemize}
        \item Verstoß gegen deutsches Recht
        \item Verstoß gegen die Europäische Menschenrechtskonvention
      \end{itemize}
    \end{frame}
    
      \subsection*{Aussagekraft von Metadaten}
    \begin{frame}<beamer>{Aussagekraft von Metadaten}
      \begin{itemize}
        \item Jakob
           \begin{itemize}
         \item Telefonierte am selben Tag: mit seiner Mutter,mit seiner Krankenkasse und mit einer AIDS-Hotline
      \end{itemize}
        \item Lisa
           \begin{itemize}
         \item Telefoniert in letzter Zeit immer weniger mit ihrem Freund. Hat aber seit längerem intensiven SMS-Kontakt zu einer neuen Nummer.
      \end{itemize}
      \end{itemize}
    \end{frame}
    
    
    

  \subsection*{Zukunft informelle Selbstbestimmung}
    \begin{frame}<beamer>{Zukunft informelle Selbstbestimmung}
      \begin{itemize}
        \item
          todo
  

      \end{itemize}
    \end{frame}

  \subsection*{Demonstrationen}
    \begin{frame}<beamer>{Demonstrationen}
      \begin{itemize}
        \item
          todo
  
      \end{itemize}
    \end{frame}
