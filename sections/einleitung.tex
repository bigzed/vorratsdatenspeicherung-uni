%!TEX root = ../main.tex

\section{Einleitung}
  \begin{frame}
    \begin{itemize}
      \frametitle{Vorratsdatenspeicherung}
      \item \textbf{Vorratsdatenspeicherung}\\
        Unter einer Vorratsdatenspeicherung (VDS) versteht man die Speicherung personenbezogener Daten durch oder für öffentliche Stellen, ohne dass die Daten aktuell benötigt werden. Sie werden also nur für den Fall gespeichert, dass sie einmal benötigt werden sollten. In der rechtspolitischen Debatte bezieht sich der Begriff meist auf die Vorratsdatenspeicherung von Telekommunikations-Verbindungsdaten.
    \end{itemize}
  \end{frame}

  \subsection*{Aussagekraft von Metadaten}
    \begin{frame}<beamer>{Aussagekraft von Metadaten}
      \begin{itemize}
        \item Jakob
           \begin{itemize}
         \item Telefonierte am selben Tag: mit seiner Mutter,mit seiner Krankenkasse und mit einer AIDS-Hotline
      \end{itemize}
        \item Lisa
           \begin{itemize}
         \item Telefoniert in letzter Zeit immer weniger mit ihrem Freund. Hat aber seit längerem intensiven SMS-Kontakt zu einer neuen Nummer.
      \end{itemize}
      \end{itemize}
    \end{frame}
