%!TEX root = ../main.tex

\section{EU-Richtlinie}
    \subsection{2006/24/EG}
    \begin{frame}
      \begin{itemize}
        \item
          Richtlinie zur Vereinheitlichung der Vorratsspeicherung von Telekomunikationsdaten
        \item
          erster Entwurf August 2002 durch d"anische Ratspr"asidentschaft
        \item
          nach Madrider Anschl"agen vom 11. M"arz 2004 offizielle Beauftragung des Ministerrats mit Pr"ufung
        \item
          29. April 2004 erster Entwurf f"ur Rahmenbeschlu"s
        \item 
          7. Juli 2005 neuer Aufschwung furch Anschl"age in London
        \item
          21. September Vorlage durch EU-Kommission
        \item
          14. Dezember 2005, 378 zu 197 Stimmen im Europaparlament, somit die schellst verabschiedete Richtlinie der EU

        % Kommentare
        % - Pruefung 2005 war ob und welche Dinge erlassen werden sollten
        % - Vorschlag von 2004 beinhaltete 36 Monate Speicherung, auch fuer Filesharing..
        % - Initiiert von Frankreich, Irland, Schweden und UK
        % - Ministerrat sieht es als seine Zustaendigkeit 'Dritte Saeule der EU'
        % - Parlament sieht es als seine 'Erste Saeule der EU'
        % - Kommision sagt beide

      \end{itemize}
    \end{frame}

    \begin{frame}
      \frametitle{2006/24/EG}
      \begin{itemize}
        \item 
          Mitgliedsstaaten haben bis zum 15. September 2007 Zeit zur Umsetzung
        \item 
          E-Mail, Internet und VoIP sind bis 15. M"arz 2009 umzusetzen
        \item
          erste Klage von Irland am 6. Juli 2006, Rechtsgrundlage mit Binnenmarktkompetenz (Artikel 95 EG) nicht ausreichend
        \item
          30. Mai 2006 Urteil zu "ubermittlung von Fluggastdaten.\\
          'EG-Rechtsakte zum Schutz der öffentlichen Sicherheit und zu Strafverfolgungszwecken sind unzulässig'
        \item 
          8. April 2014 erkl"art der Europ"aische Gerichtshof die Richtlinie f"ur Ung"ultig, versto"st gegen die Charta der Grundrechte der Europ"aischen Union
      \end{itemize}
    \end{frame}

    \begin{frame}
      \frametitle{R"uchverfolgng und Identifizierung der Quelle/des Empf"angers}
      \begin{itemize}
        \item Telefonnetz
        \begin{itemize}
          \item Art des Vorgangs
          \item Rufnummer des Anschlusses
          \item Name und Anschrift des Teilnehmers
        \end{itemize}
        \item Internet
        \begin{itemize}
          \item Art des Vorgangs
          \item Benutzerkennung
          \item Name und Anschrift des Teilnehmers
        \end{itemize}
      \end{itemize}
    \end{frame}

    \begin{frame}
      \frametitle{Bestimmung von Datum, Uhrzeit, Dauer einer Kommunikation}
      \begin{itemize}
        \item Telefonnetz
        \begin{itemize}
          \item Datum und Uhrzeit zu Beginn und Ende des Vorgangs
        \end{itemize}
        \item Internet
        \begin{itemize}
          \item Datum und Uhrzeit der An- und Abmeldung
          \item zugeh"orige IP-Adresse
          \item Benutzerkennung des Nutzers
          \item Datum und Uhrzeit der An- und Abmeldung bei E-Mail/VoIP-Diensten
        \end{itemize}
      \end{itemize}
    \end{frame}

    \begin{frame}
      \frametitle{Bestimmung der Endeinrichtung}
      \begin{itemize}
        \item Telefonnetz
        \begin{itemize}
          \item internationale Mobilteilnehmerkennung (IMSI)
          \item internationale Mobilfunkger"atekennung (IMEI)
          \item bei annonymen Diensten Datum, Uhrzeit und Cell-ID der Aktivierung
        \end{itemize}
      \end{itemize}
    \end{frame}

    \begin{frame}
      \frametitle{Standortbestimmung mobiler Endger"ate}
      \begin{itemize}
        \item Cell-ID bei Beginn der Verbindung
        \item Daten zur geographischen Ortung von Funkzellen w"ahrend der Kommunikation
      \end{itemize}
    \end{frame}